\documentclass[11pt]{article}

\usepackage[english,italian]{babel}
\usepackage[a4paper, top=2cm, bottom=1.5cm, left=2cm, right=2cm]{geometry}
\usepackage{float}
\usepackage{ltablex}
\usepackage{titling}
\usepackage{blindtext}
\usepackage[utf8]{inputenc}
\usepackage[T1]{fontenc}
\usepackage{xcolor}
\usepackage{graphicx}
\usepackage{geometry}
\usepackage[italian]{babel}
\usepackage{tabularx}
\usepackage{longtable}
\usepackage{hyperref}
\usepackage[bottom]{footmisc}
\usepackage{fancyhdr}
\usepackage{titlesec}
\setcounter{secnumdepth}{4}
\usepackage{amsmath, amssymb}
\usepackage{array}
\usepackage{graphicx}
\usepackage{layouts}
\usepackage{url}
\usepackage{comment}
\usepackage{eurosym}


\begin{document}

	\thispagestyle{empty}
	\begin{titlepage}
		\begin{center}
			\includegraphics[scale = 0.05]{../../Res/logo_unipd.png}\\
			\bigskip
			\large \textbf{Università degli Studi di Padova} \\
			\vfill
			\includegraphics[scale = 0.7]{../../Res/BugPharma_Logo.png}\\
			\huge \textbf{Gruppo Bug Pharma} \\
			\vfill
			\large \texttt{bugpharma10@gmail.com}
			\vfill
			\Huge \textbf{Verbale esterno del 15-11-2021}\\
			
			\large
			\textbf{Zucchetti S.p.A. - Login Warrior}\\
			
			\vfill
			
			\begin{tabular}{r|l}
				\textbf{Approvazione} &  -\\
				\textbf{Redazione} &  \parbox[t]{3.5cm}{Andrea Salmaso \\Nicla Faccioli}\\
				\textbf{Verifica} &  -\\
				\textbf{Stato} & Redatto \\
				\textbf{Uso} & Esterno
			\end{tabular}
			\vfill
			
		\end{center}
	\end{titlepage}

	\newpage

	\section{Informazioni generali}
	\subsection{Luogo e data dell'incontro}
	\begin{itemize}
		\item \textbf{Luogo}: videoconferenza Zoom;
		\item \textbf{Data}: 15-11-2021;
		\item \textbf{Ora di inizio}: 16:30;
		\item \textbf{Ora di fine}: 17:00.
	\end{itemize}
	
	\subsection{Presenze}
	\begin{itemize}
		\item \textbf{Totale presenze}: 7;
		\item \textbf{Presenti}:
		\begin{itemize}
			\item Lorenzo Piran; 
			\item Michele Masetto;
			\item Silvia Giro;
			\item Nicla Faccioli (segretaria);
			\item Sara Nanni;
			\item Andrea Salmaso;
			\item Nicholas Sertori.
		\end{itemize}
		\item \textbf{Assenti}: nessuno.
		\item \textbf{Partecipanti esterni}: 
			\begin{itemize}
				\item Gregorio Piccoli (Zucchetti S.p.A.).
			\end{itemize}
	\end{itemize}

	\newpage

	\section{Domande e risposte}
		\begin{enumerate}
			\item \textbf{Per chi è pensato il sito/progetto? Per uso interno all'azienda Zucchetti o è destinato
			ad essere distribuito anche ad altre aziende?} 
			
			\medskip
			
			\begin{itemize}
				\item È complicato fare un prodotto completo, l'idea principale è:
				\begin{itemize}
					\item Poter caricare i dati contenuti nel file \texttt{CSV};
					\item Creare qualcosa che possa essere usato dal data analyst per controllare i risultati del monitoraggio
					e valutare se le cose stanno andando bene o no.
				\end{itemize}
				\item È stata mostrata dal proponente una dashboard di monitoraggio del sistema di fatturazione elettronica
				attualmente in uso all'interno dell'azienda:
				\begin{itemize}
					\item Nel visionarla, il sistemista, in caso di attività sospetta, ha la possibilità di accedere ad una
					dashboard più specifica	per visualizzarne i dettagli e comprendere meglio la situazione.
				\end{itemize}
				\item Il sistema da progettare diventerebbe un'integrazione al sistema già esistente per il controllo dei dati
				di login già ottenuti da altre fonti;
				\item È stata fatta notare dal proponente l'esistenza di due tipi di dati di login:
				\begin{itemize}
					\item I dati storici, da utilizzare come riferimento;
					\item I dati attuali, da mixare ai dati storici.
				\end{itemize}
				È necessario confrontare i dati attuali con i dati storici per comprendere il contesto (es: gli orari potrebbero
				cambiare in base al paese e al fuso orario).
			\end{itemize}
			
			\bigskip
			
			\item \textbf{Quindi il nostro prodotto verrà usato solo da utenti specifici, ovvero questi analisti che si occupano
			di analizzare i dati?}
			
			\medskip
			
			\begin{itemize}
				\item L'utenza principale sarà costituita da analisti;
				\item Non bisogna fare assunzioni sulla loro preparazione: potrebbero essere anche dipendenti che stanno imparando;
				\item È comunque necessario fare un prodotto usabile (es: no riga di comando).
			\end{itemize}
			
			\bigskip 
			
			\item \textbf{Bisogna sviluppare un Applicativo Web. Essendo probabilmente ad uso interno, è prevedibile
			l'implementazione di un sistema di login?}
			
			\medskip
			\begin{itemize}
				\item L'argomento sicurezza non costituisce il tema centrale del progetto;
				\item Aspetto troppo ampio, è conveniente non trattarlo per una questione di tempo;
				\item Il tema da sviluppare è l'analisi dei dati.
			\end{itemize}
			  
			\bigskip
			
			\item \textbf{Per la costruzione del sito ci si può basare su qualche framework già essitente?}
			
			\medskip
			\begin{itemize}
				\item La decisione spetta al gruppo;
				\item È possibile utilizzare lo strumento che si preferisce, la questione è però di tipo diverso:
				\begin{itemize}
					\item Ai fini del progetto è considerabile l'utilizzo di un framework qualunque;
					\item Concretamente, spesso tali framework hanno una dura di vita inferiore a quella di applicazioni
					gestionali come quella oggetto del capitolato (che può giungere e superare i vent'anni):
					\begin{itemize}
						\item È necessario un fortissimo controllo per evitare un enorme debito tecnico in futuro.
					\end{itemize}
				\end{itemize}
				\item Nel caso in cui il gruppo decidesse di farne uso:
				\begin{itemize}
					\item Valutare \textit{jQuery}, abbastanza neutro;
					\item \textit{React} e \textit{Angular} risulterebbero essere troppo impattanti, costringendo a
					seguire "la loro volontà";
					\item Consigliabile usarne di già familiari ai membri del gruppo, per evitare di impiegare troppo
					tempo al loro studio;
					\item Consigliabile usarne di conosciuti, con una buona documentazione:
					\begin{itemize}
						\item \textit{React};
						\item \textit{VUE};
						\item \textit{Angular};
						\item \textit{Svelte}.
					\end{itemize}
				\end{itemize}
			\end{itemize}
			
			\bigskip
			
			\item \textbf{Le librerie fornite per la riduzione dimensionale in che linguaggio sono scritte? Vengono applicate
			lato server?}
			
			\medskip
			
			\begin{itemize}
				\item Sono tutte scritte in \textit{JavaScript} e vengono applicate lato browser;
				\item Per la riduzione dimensionale è possibile usare:
				\begin{itemize}
					\item La libreria fornita dall'azienda;
					\item \textit{DruidJS} (utilizzata nel progetto proposto lo scorso anno).
				\end{itemize}
			\end{itemize}
			
			\bigskip
			
			\item \textbf{Nel caso avessimo bisogno di consigli per quanto riguarda la libreria \textit{D3.js} o gli
			algoritmi di ML, a chi sarebbe utile rivolgersi?}
			
			\medskip
			
			È possibile rivolgersi sempre al referente di Zucchetti S.p.A., Gregorio Piccoli.
			
			\bigskip
			
			\item \textbf{Quale potrebbe essere una durata approssimativa di un progetto di questo tipo o una possibile data di
			fine approssimativa? }
			
			\medskip
			
			\begin{itemize}
				\item In genere la data di fine progetto ricade verso fine Aprile/Maggio, anche se quest'anno sono cambiate le cose;
				\item Una stima ragionevole potrebbe essere per i primi di Maggio, dipende molto dal gruppo.
			\end{itemize}

		\end{enumerate}
		
\end{document}