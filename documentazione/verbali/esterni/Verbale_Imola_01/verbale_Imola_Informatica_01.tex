\documentclass[11pt]{article}

\usepackage[english,italian]{babel}
\usepackage[a4paper, top=2cm, bottom=1.5cm, left=2cm, right=2cm]{geometry}
\usepackage{float}
\usepackage{ltablex}
\usepackage{titling}
\usepackage{blindtext}
\usepackage[utf8]{inputenc}
\usepackage[T1]{fontenc}
\usepackage{xcolor}
\usepackage{graphicx}
\usepackage{geometry}
\usepackage[italian]{babel}
\usepackage{tabularx}
\usepackage{longtable}
\usepackage{hyperref}
\usepackage[bottom]{footmisc}
\usepackage{fancyhdr}
\usepackage{titlesec}
\setcounter{secnumdepth}{4}
\usepackage{amsmath, amssymb}
\usepackage{array}
\usepackage{graphicx}
\usepackage{layouts}
\usepackage{url}
\usepackage{comment}
\usepackage{eurosym}


\begin{document}

	\thispagestyle{empty}
	\begin{titlepage}
		\begin{center}
			\includegraphics[scale = 0.05]{../../Res/logo_unipd.png}\\
			\bigskip
			\large \textbf{Università degli Studi di Padova} \\
			\vfill
			\includegraphics[scale = 0.7]{../../Res/BugPharma_Logo.png}\\
			\huge \textbf{Gruppo Bug Pharma} \\
			\vfill
			\large \texttt{bugpharma10@gmail.com}
			\vfill
			\Huge \textbf{Verbale esterno del 16-11-2021}\\
			
			\large
			\textbf{Imola Informatica S.p.A. - Bot4Me}\\
			
			\vfill
			
			\begin{tabular}{r|l}
				\textbf{Approvazione} &  Nicholas Sertori\\
				\textbf{Redazione} &  \parbox[t]{3.5cm}{Andrea Salmaso \\Nicla Faccioli}\\
				\textbf{Verifica} &  Silvia Giro\\
				\textbf{Stato} & Approvato \\
				\textbf{Uso} & Esterno
			\end{tabular}
			\vfill
			
		\end{center}
	\end{titlepage}

	\newpage

	\section{Informazioni generali}
	\subsection{Luogo e data dell'incontro}
	\begin{itemize}
		\item \textbf{Luogo}: videoconferenza Microsoft Teams;
		\item \textbf{Data}: 16-11-2021;
		\item \textbf{Ora di inizio}: 16:00;
		\item \textbf{Ora di fine}: 16:30.
	\end{itemize}
	
	\subsection{Presenze}
	\begin{itemize}
		\item \textbf{Totale presenze}: 5 su 7;
		\item \textbf{Presenti}:
		\begin{itemize}  
			\item Michele Masetto;
			\item Nicla Faccioli;
			\item Sara Nanni;
			\item Andrea Salmaso;
			\item Nicholas Sertori.
		\end{itemize}
		\item \textbf{Assenti}:
			\begin{itemize}
				\item Lorenzo Piran;
				\item Silvia Giro.
			\end{itemize}
		\item \textbf{Partecipanti esterni}:
			\begin{itemize}
				\item Alessandro Proscia (Imola Informatica S.p.A.);
				\item Giacomo Lorenzo (Imola Informatica S.p.A.).
			\end{itemize}
	\end{itemize}

	\newpage

	\section{Domande e risposte}
		\begin{enumerate}
			\item \textbf{Il resoconto delle attività fatto dai dipendenti è vincolato dal timbro di entrata? Deve
			essere effettuato obbligatoriamente prima del timbro di uscita?}
			
			\medskip
			
			\begin{itemize}
				\item Dipende molto da quanto è diligente il dipendente:
				\begin{itemize}
					\item Non si tratta di un'attività giornaliera, ma ogni dipendente può effettuarla con cadenze differenti,
					a propria discrezione;
					\item Dipende dalle regole dell'azienda: ad Imola Informatica non vi è un orario di ingresso o un cartellino
					da timbrare, per cui non vi sono obblighi in questo senso.
				\end{itemize}
			\end{itemize}

			\bigskip
			
			\item \textbf{Per quanto riguarda la gestione della presenza in sede e il tracciamento delle presenze ci sono
			limitazioni?}
			
			\medskip
			
			L'azienda utilizza un'applicazione per la consuntivazione completamente auto-contenuta al cui interno:
			\begin{itemize}
				\item Sono presenti le attività giornaliere su cui si può consuntivare;
				\item È possibile effettuare l'operazione di check-in in cui si può specificare anche la sede in cui
				si è presenti al momento.
			\end{itemize}
			Quindi il tracciamento è un'attività già inclusa nella suddetta.
			
			\bigskip 
			
			\item \textbf{Quindi il bot che dobbiamo creare si appoggerebbe su questo sistema già esistente?}
			
			\medskip
			
			\begin{itemize}
				\item L'azienda fornirà un set di API \textit{REST} definite mediante un file \textit{Swagger};
				\item È stato illustrato brevemente dal proponente il funzionamento di un file \textit{Swagger}:
				\begin{itemize}
					\item È un linguaggio che permette la definizione di API \textit{REST}, fornendo i formalismi per la
					definizione	dei servizi da esporre utilizzando il paradigma \textit{REST} (path esposti, metodi e così via);
					\item Un file \textit{Swagger} si divide in tre parti: 
					\begin{itemize}
						\item Una parte di metadata atta ad indicare:
						\begin{itemize}
							\item Quali sono le macro categorie di servizi esposti, attraverso una loro descrizione;
							\item Quali sono le licenze con cui si stanno esponendo le suddette API
							(\textit{http} o \textit{https}, base path ecc.).
						\end{itemize}
						\item La definizione dei vari servizi esposti:
						\begin{itemize}
							\item Ogni servizio è esposto su un path ed ha associati dei metodi \textit{http};
							\item \textit{REST} indica che è possibile accedere a diverse risorse, ciascuna delle quali ha
							associate alcune operazioni;
							\item  Ogni operazione restituisce un \texttt{json} di risposta in cui sono contenute informazioni
							rilevanti per il tipo di operazione;
							\item Ogni variazione sul path identifica una risorsa differente.
						\end{itemize}												
						\item Una parte di definizione degli oggetti usati dalle varie API:
						\begin{itemize}
							\item Ogni oggetto avrà una serie di caratteristiche.
						\end{itemize}												
					\end{itemize} 
				\end{itemize}
			\end{itemize}
			
			
			
			\bigskip 
			
			\item \textbf{Per quanto riguarda il requisito opzionale sull'accesso ai documenti: tale accesso dipende dai
			privilegi del dipendente (e quindi dal suo ruolo all'interno dell'azienda) oppure è una cosa più generica in cui tutti
			hanno accesso a tutto?}
			
			\medskip
			
			\begin{itemize}
				\item In azienda viene già utilizzato \textit{Alfresco} un sistema documentale in cui l'utente ha associati
				alcuni ruoli;
				\item A seconda del ruolo l'utente vede o non vede alcune cose;
				\item Per quanto riguarda \textit{Alfresco}:
				\begin{itemize}
					\item Lo standard \textit{CIVIS} si occupa della standardizzazione totale dell'accesso documentale;
					\item È possibile trovare librerie già pronte che potrebbero tornare utili (es.: \textit{Apache Chemistry}
					per \textit{Java}.
				\end{itemize}
				\item Per quanto riguarda le scelte tecnologiche, il gruppo è libero di scegliere ciò che vuole.
			\end{itemize}
			
			\bigskip 
			
			\item \textbf{L'azienda ha qualche preferenza sul tipo di formato dei documenti da consegnare?}
			
			\medskip
			
			\begin{itemize}
				\item Non ci sono preferenze per quanto riguarda il formato;
				\item A livello di progetto software, oltre al codice i documenti solitamente consegnati sono:
				\begin{itemize}
					\item Analisi funzionale: per descrivere a livello funzionale qual è il progetto, quali sono i requisiti e come funziona;
					\item Analisi tecnica: per descrivere in dettaglio le parti tecniche;
					\item Macro schema architetturale: per descrivere com'è fatta l'applicazione;
					\item Analisi degli use case;
					\item Manuale d'uso.
				\end{itemize}
			\end{itemize}

			\bigskip
			
			\item \textbf{Come si gestisce un errore nell'interazione con l'utente come un'incomprensione del messaggio?}
			
			\medskip
			
			\begin{itemize}
				\item Viene restituito all'utente un generico messaggio di errore (come fanno \textit{Alexa} o \textit{Google Assistant});
				\item Non è necessaria compensazione (rollback del sistema per riportarlo allo stato originale) in caso di errori.
			\end{itemize}

			\bigskip 
			
			\item \textbf{Consigli o aspettative da questa collaborazione?}
			
			\medskip

			\begin{itemize}
				\item La risposta è stata riformulata dal proponente in: "Cosa deve aspettarsi il gruppo da Imola?":
				\begin{itemize}
					\item Verrà messo a disposizione un canale diretto di comunicazione per chiarire dubbi o perplessità (anche su
					\textit{Telegram});
					\item L'azienda rimane disponibile per meeting periodici in stile \textit{scrum}, anche per eventuali dubbi tecnici
					o nel caso in cui il gruppo si trovasse bloccato su qualche aspetto.
				\end{itemize}
				\item Dall'altro lato, il proponente si aspetta di vedere un prodotto funzionante.
				\item Sono state fornite altre generiche informazioni sul chatbot:
				\begin{itemize}
					\item Usare le tecnologie che si ritengono più opportune;
					\item Cercare materiale in autonomia, facilmente recuperabile.
				\end{itemize}
				\item Sono stati forniti alcuni collegamenti utili a varie risorse:
				\begin{itemize}
					\item Strumenti:
					\begin{itemize}
						\item \href{https://editor.swagger.io/}{\texttt{Swagger}};
						\item \href{https://chemistry.apache.org/}{\texttt{Apache Chemistry}};
						\item \href{https://www.datacamp.com/community/tutorials/building-a-chatbot-using-chatterbot}{\texttt{Chatterbot}};
					\end{itemize}
					\item Contatti:
					\begin{itemize}
						\item \href{https://t.me/aleproscia}{\texttt{Alessandro Poscia (Telegram)}};
						\item \href{https://t.me/giacomolorenzo}{\texttt{Giacomo Lorenzo (Telegram)}}.
					\end{itemize}
				\end{itemize}
			\end{itemize}
		\end{enumerate}
		
\end{document}
