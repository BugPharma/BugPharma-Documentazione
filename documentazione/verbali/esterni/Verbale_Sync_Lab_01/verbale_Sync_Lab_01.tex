\documentclass[11pt]{article}

\usepackage[english,italian]{babel}
\usepackage[a4paper, top=2cm, bottom=2cm, left=2cm, right=2cm]{geometry}
\usepackage{float}
\usepackage{ltablex}
\usepackage{titling}
\usepackage{blindtext}
\usepackage[utf8]{inputenc}
\usepackage[T1]{fontenc}
\usepackage{xcolor}
\usepackage{graphicx}
\usepackage{geometry}
\usepackage[italian]{babel}
\usepackage{tabularx}
\usepackage{longtable}
\usepackage{hyperref}
\usepackage[bottom]{footmisc}
\usepackage{fancyhdr}
\usepackage{titlesec}
\setcounter{secnumdepth}{4}
\usepackage{amsmath, amssymb}
\usepackage{array}
\usepackage{graphicx}
\usepackage{layouts}
\usepackage{url}
\usepackage{comment}
\usepackage{eurosym}


\begin{document}

	\thispagestyle{empty}
	\begin{titlepage}
		\begin{center}
			\includegraphics[scale = 0.05]{../../Res/logo_unipd.png}\\
			\bigskip
			\large \textbf{Università degli Studi di Padova} \\
			\vfill
			\includegraphics[scale = 0.7]{../../Res/BugPharma_Logo.png}\\
			\huge \textbf{Gruppo Bug Pharma} \\
			\vfill
			\large \texttt{bugpharma10@gmail.com}
			\vfill
			\Huge \textbf{Verbale esterno del 12-11-2021}\\
			
			\large
			
			
			\vfill
			
			\begin{tabular}{r|l}
				\textbf{Approvazione} &  -\\
				\textbf{Redazione} &  \parbox[t]{3.5cm}{Andrea Salmaso \\Nicla Faccioli}\\
				\textbf{Verifica} &  -\\
				\textbf{Stato} & Redatto \\
				\textbf{Uso} & Esterno
			\end{tabular}
			\vfill
			
		\end{center}
	\end{titlepage}

	\newpage
	
	\section{Informazioni generali}
		\subsection{Luogo e data dell'incontro}
			\begin{itemize}
				\item \textbf{Luogo}: videoconferenza Google Meet;
				\item \textbf{Data}: 12-11-2021;
				\item \textbf{Ora di inizio}: 16:00;
				\item \textbf{Ora di fine}: 16:40.
			\end{itemize}
		
		\subsection{Presenze}
			\begin{itemize}
				\item \textbf{Totale presenze}: 5 su 7;
				\item \textbf{Presenti}:
				\begin{itemize}
					\item Lorenzo Piran; 
					\item Michele Masetto;
					\item Nicla Faccioli (segretaria);
					\item Andrea Salmaso;
					\item Nicholas Sertori.
				\end{itemize}
				\item \textbf{Assenti}: 
					\begin{itemize}
						\item Silvia Giro;
						\item Sara Nanni.
					\end{itemize}
				\item \textbf{Partecipanti esterni}:
				\begin{itemize}
					\item Fabio Pallaro (Sync Lab S.r.l.).
				\end{itemize}
			\end{itemize}
		
		\newpage
		
		\section{Domande e risposte}
			\begin{enumerate}
				\item \textbf{Cosa succederebbe se ci fosse un pacco smarrito (es: problema di logistica, anche se il venditore
				ha inviato il pacco e l'acquirente ha pagato - soldi bloccati nello smart contract)?}
				
				\medskip		
						
				Viene consigliato di prendere spunto da \textit{PayPal} o simili: 
				\begin{enumerate}
					\item L'acquirente apre una contestazione da app mobile oppure con interfaccia lato cliente
					(autenticandosi con il proprio wallet);
					\item Vede le sue transazioni in corso;
					\item Per la transazione di cui non ha ricevuto il pacco potrà segnalare il problema.
					\item All'apertura della segnalazione, arriverà una notifica al venditore (invio alla mail del venditore
					registrata sul database);
					\item Il venditore analizza il caso e nascono due possibilità: 
						\begin{itemize}
							\item Il venditore annulla la transazione (comunica con la blockchain e i soldi tornano all'acquirente); 
							\item In caso di arrivo del pacco, il cliente annulla la contestazione e i soldi arrivano
							al venditore.
						\end{itemize}
				\end{enumerate}
				
				\bigskip
				
				\item \textbf{E se il pacco arriva dopo aver concluso la segnalazione e aver restituito i soldi?}
				
				\medskip
				
				\begin{itemize}
					\item Si può prevenire il problema lasciando passare un paio di mesi prima di effettuare il rimborso,
					in modo da permettere al pacco di arrivare/essere restituito al mittente.
					\item Se il pacco va perso, il problema viene rimandato al corriere (i soldi tornano all'acquirente e il
					venditore viene risarcito con l'assicurazione della compagnia di trasporti).
				\end{itemize}
				
				\bigskip				
				
				\item \textbf{Serve anonimato? È possibile ricontattare il venditore, per natura stessa della blockchain?}
				
				\medskip
				
				\begin{itemize}
					\item Sia venditore che acquirente sono in chiaro. Infatti:
					\begin{itemize}
						\item Il cliente che compra va sul portale web del venditore 
						\item Il sito di e-commerce conosce nome, cognome e indirizzo di consegna dell'acquirente dalla
						registrazione al sito, a prescindere dal servizio.
					\end{itemize}
					\item Nel caso in esame, quindi, si può tenere tutto in chiaro, anche eventuali comunicazioni tra le due
					parti:
					\begin{itemize}
						\item Si esclude la possibilità di dare all'acquirente un modo di comunicare direttamente con il servizio;
						\item Tale possibilità verrà comunque fornita al venditore (che dovrà iscriversi alla piattaforma per
						poterla sfruttare).
					\end{itemize}
					\item \textbf{OPZIONALE:} si può pensare di implementare una web-app di monitoraggio delle transazioni
					per l'acquirente;
					\item Il gruppo aveva pensato ad una soluzione in stile \textit{Amazon-Locker} per garantire una sorta di anonimato per
					l'acquirente, ma è preferibile rimanere sul semplice e fare tutto in chiaro.
				\end{itemize}
				
				\bigskip
				
				\item \textbf{Che politica si applica per resi e rimborsi? E come si può esser sicuri che il venditore, una volta
				ricevuto il pagamento, non scappi con i soldi (L'acquirente paga ma il pacco non contiene quello che dovrebbe -
				Problema del mattone)?}
				
				\medskip
				
				\begin{itemize}
					\item \textbf{OPZIONALE:} Si potrebbe procedere nel seguente modo:
					\begin{enumerate}
						\item Nel momento dello sblocco del pacco non viene rilasciato tutto l'importo ma solo metà;
						\item Una volta aperto il pacco e verificato che contiene ciò che deve, viene rilasciato un altro
						quarto dell'importo totale;
						\item Quando l'acquirente ha verificato che sia tutto regolare, viene rilasciato l'ultimo quarto.
					\end{enumerate}
					\item Potrebbero comunque esserci inconvenienti: ad esempio se l'acquirente sblocca la prima metà e poi non sblocca
					più niente, i soldi restano fermi sullo smart-contract.
					\begin{itemize}
						\item \underline{Opzione interessante:} dopo qualche settimana avviene uno sblocco automatico del resto dei soldi
						nel caso in cui non vengano sollevate segnalazioni da parte dell'acquirente (le tempistiche non sono un
						problema, sarà tutto configurabile).
					\end{itemize}
				\end{itemize}
				
				\bigskip
				
				\item \textbf{Come confermare la ricezione? (Ad esempio, se si utilizza la scansione del \textit{QR code}, cosa
				succederebbe se all'arrivo del pacco l'acquirente non avesse la possibilità di fare lo scan per qualche motivo)}
				
				\medskip
				
				Oltre che usare il \textit{QR code}, si deve fornire un secondo metodo per la conferma della ricezione, sfruttando il fatto
				che il \textit{QR code} è in realtà una stringa: nell'applicazione potrebbe esserci la possibilità di digitare manualmente
				tale stringa (riportata anche sul pacco).
				
				\bigskip
				
				\item \textbf{E se si volesse delegare qualcuno per ritirare il pacco?}
				
				\medskip
				
				\begin{itemize}
					\item Punto debole: di base il pacco va sbloccato con il wallet dell'acquirente;
					\item \textbf{OPZIONALE:} Come soluzione si potrebbe creare un meccanismo per dare all'acquirente la possibilità di indicare
					dei delegati, specificando i loro wallet address.
				\end{itemize}

				\bigskip
				
				\item \textbf{I componenti del gruppo hanno poca (se non nulla) esperienza con blockchain e cripto valute.
				Con la scelta di questo capitolato dovrebbero studiare queste tecnologie da zero.
				Quanto è necessario approfondire il tema per poter presentare un progetto adeguato?
				In altre parole: è fattibile scegliere questo capitolato senza avere conoscenze pregresse sulle tecnologie
				che sembrano essere un po' il punto centrale?}
				
				\medskip
				
				\begin{itemize}
					\item È possibile tarare il progetto sulle conoscenze dei membri del gruppo, in modo da lavorare sulla parte blockchain
					il minimo indispensabile (usando possibilmente \textit{Ethereum}, avente una grande community, e sfruttando
					materiali forniti dall'azienda);
					\item Non sono presenti particolari problemi: per bilanciare il gruppo si potrebbe concentrare di più sulla
					parte acquirente, dato che la parte blockchain sarebbe un po' più povera.
				\end{itemize}
			\end{enumerate}
\end{document}