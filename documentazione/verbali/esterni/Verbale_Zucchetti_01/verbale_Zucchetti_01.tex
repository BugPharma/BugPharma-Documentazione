\documentclass[11pt]{article}

\usepackage[english,italian]{babel}
\usepackage[a4paper, top=2cm, bottom=1.5cm, left=2cm, right=2cm]{geometry}
\usepackage{float}
\usepackage{ltablex}
\usepackage{titling}
\usepackage{blindtext}
\usepackage[utf8]{inputenc}
\usepackage[T1]{fontenc}
\usepackage{xcolor}
\usepackage{graphicx}
\usepackage{geometry}
\usepackage[italian]{babel}
\usepackage{tabularx}
\usepackage{longtable}
\usepackage{hyperref}
\usepackage[bottom]{footmisc}
\usepackage{fancyhdr}
\usepackage{titlesec}
\setcounter{secnumdepth}{4}
\usepackage{amsmath, amssymb}
\usepackage{array}
\usepackage{graphicx}
\usepackage{layouts}
\usepackage{url}
\usepackage{comment}
\usepackage{eurosym}


\begin{document}

	\thispagestyle{empty}
	\begin{titlepage}
		\begin{center}
			\includegraphics[scale = 0.05]{../../Res/logo_unipd.png}\\
			\bigskip
			\large \textbf{Università degli Studi di Padova} \\
			\vfill
			\includegraphics[scale = 0.7]{../../Res/BugPharma_Logo.png}\\
			\huge \textbf{Gruppo Bug Pharma} \\
			\vfill
			\large \texttt{bugpharma10@gmail.com}
			\vfill
			\Huge \textbf{Verbale esterno del 05-11-2021}\\
			
			\large
			\textbf{Zucchetti S.p.A. - Login Warrior}\\
			
			\vfill
			
			\begin{tabular}{r|l}
				\textbf{Approvazione} &  Nicholas Sertori\\
				\textbf{Redazione} &  \parbox[t]{3.5cm}{Andrea Salmaso \\Nicla Faccioli}\\
				\textbf{Verifica} &  Silvia Giro\\
				\textbf{Stato} & Approvato \\
				\textbf{Uso} & Esterno
			\end{tabular}
			\vfill
			
		\end{center}
	\end{titlepage}

	\newpage

	\section{Informazioni generali}
	\subsection{Luogo e data dell'incontro}
	\begin{itemize}
		\item \textbf{Luogo}: videoconferenza Zoom;
		\item \textbf{Data}: 05-11-2021;
		\item \textbf{Ora di inizio}: 16:30;
		\item \textbf{Ora di fine}: 17:00.
	\end{itemize}
	
	\subsection{Presenze}
	\begin{itemize}
		\item \textbf{Totale presenze}: 7;
		\item \textbf{Presenti}:
		\begin{itemize}
			\item Lorenzo Piran;
			\item Michele Masetto;
			\item Silvia Giro;
			\item Nicla Faccioli;
			\item Sara Nanni;
			\item Andrea Salmaso;
			\item Nicholas Sertori.
		\end{itemize}
		\item \textbf{Assenti}: nessuno.
		\item \textbf{Partecipanti esterni}:
		\begin{itemize}
			\item Gregorio Piccoli (Zucchetti S.p.A.).
		\end{itemize}
	\end{itemize}

	\newpage

	\section{Domande e risposte}
		\begin{enumerate}
			\item \textbf{Che tipo di sito serve?
			C'è altro oltre alla visualizzazione dei dati, come ad esempio il caricamento dei dati da parte dell'utente?
			Che livello di interazione con il sito serve da parte dell'utente?}
			\medskip
			\begin{itemize}
				\item Il caricamento dei dati tramite \texttt{CSV} dovrebbe stare nell'applicazione.
				\item Il tema principale è la visualizzazione:
				\begin{itemize}
					\item [(a)] Con la libreria \textit{D3.js} vengono generati grafici a partire dai dati caricati;
					\item [(b)] Successivamente si va a discrezione del fornitore.
				\end{itemize}
				Centrale quindi l'interazione con l'utente.
			\end{itemize}
		
			\bigskip
			
			\item \textbf{Qual è il punto centrale del progetto?
			Solo la preparazione e l'analisi dei dati o dobbiamo preoccuparci anche della sicurezza?}
			\medskip
			\begin{itemize}
				\item Idea principale: utilizzo del prodotto da parte di un sistemista che esegue un monitoraggio
				dei dati di login a vari applicativi aziendali da parte dei dipendenti.
				\begin{itemize}
					\item [(a)] I dati di login vengono forniti in un file \texttt{CSV};
					\item [(b)] Il file viene caricato nel sistema per l'esplorazione dei dati;
					\item [(c)] Il sistema visualizza vari grafici (i quali devono essere scelti dal fornitore).
				\end{itemize}
				\item Per quanto riguarda la sicurezza dei dati:
				\begin{itemize}
					\item Se il file contenente i dati viene caricato in locale: è un tema marginale;
					\item Se il file viene caricato nel server per una più facile gestione (data la quantità di dati):
					è necessario valutarla.
					Per quanto sensata sia come scelta, dal punto di vista del progetto potrebbe portare via tempo a
					cose più interessanti (come il Machine Learning).\\
					\underline{Idea:} Si potrebbe segnalare al  committente, il quale saprà di dover aumentare il budget.
				\end{itemize}
			\end{itemize}

			\bigskip
			
			\item \textbf{Che tipo di dati ci sono nei file \texttt{CSV}?
			Consigli su come utilizzarli?}
			\medskip
			\begin{itemize}
				\item È stato aperto un file di esempio e ne sono stati mostrati i punti salienti.\\
				I dati che l'azienda ci fornirà contengono:
				\begin{itemize}
					\item Numero utente;
					\item Data dell'evento (dato \textbf{importante});
					\item Tipo di evento:
					\begin{itemize}
						\item Login;
						\item Errore login;
						\item Logout.
					\end{itemize}
					\item Applicativo di accesso;
					\item IP sorgente;
					\item Altri dati non necessari.
				\end{itemize}
				\item Conviene dividere la data nelle sue parti (giorno della settimana, mese, settimana dell'anno, ecc)
				per sfruttarla meglio, così come la fascia oraria (ore e minuti).
				\item \textbf{Obiettivo finale}: vedere come si distribuiscono.
			\end{itemize}
			
			\bigskip
			
			\item \textbf{Quali sono i parametri per giudicare un login sospetto?}
			\medskip
			\begin{itemize}
				\item Nessun grafico potrà supportare tutte le coordinate contenute nel file \texttt{CSV};
				\item Problemi centrali:
				\begin{itemize}
					\item Scelta delle coordinate da mettere nel grafico;
					\item Filtrazione dei dati per ricavarne la massima utilità.
				\end{itemize}
				\item Sarà necessario valutare il comportamento, a livello prestazionale, del grafico, data la mole di
				dati che dovranno essere visualizzati.\\
				Sono stati dati diversi spunti:
				\begin{itemize}
					\item Visualizzare in sovrimpressione solo i dati di un utente, lasciando sullo sfondo una
					nebbiolina con i dati degli altri utenti;
					\item Utilizzare qualche algoritmo/libreria di ML, fornito/a dall'azienda (es: \textit{PCA}, \textit{t-SNE},
					\textit{UMAP}) per riduzione dimensionale (ridurre il numero di coordinate):
					\begin{itemize}
						\item \textit{PCA}: riduzione lineare;
						\item \textit{t-SNE}, \textit{UMAP}: non lineare, preservano le strutture locali a scapito della
						disposizione generale (esaltazione della distanza).
					\end{itemize}
				\end{itemize}	
				\item Dopo aver ottenuto una visualizzazione ottimale delle informazioni si può considerare l'introduzione
				di algoritmi di ML che capiscano quello che l'occhio umano riesce a vedere (clusterizzazione, novelty detection,
				domain approximation), come ad esempio \textit{DBScan}, \textit{HDBScan}, \textit{One-Class SVM}.
			\end{itemize}

			\bigskip
			
			\item \textbf{Che tipo di formato preferiscono per i manuali da consegnare (utilizzo ed espansione)?}
			
			\medskip
			
			È possibile scrivere i manuali in markdown e usare il software \textit{Pandoc} per tradurli
			in formato \textit{pdf}, \textit{LaTeX} o altro.
			
			\bigskip
			
			\item \textbf{Consigli generali relativi agli anni precedenti?}
			
			\medskip
			
			Attenersi alle tecnologie che sono state indicate come consigliate.
			
		\end{enumerate}
\end{document}
