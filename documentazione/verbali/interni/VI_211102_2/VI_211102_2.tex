\documentclass[11pt]{article}

\usepackage[english,italian]{babel}
\usepackage[a4paper, top=2cm, bottom=1.5cm, left=2cm, right=2cm]{geometry}
\usepackage{float}
\usepackage{ltablex}
\usepackage{titling}
\usepackage{blindtext}
\usepackage[utf8]{inputenc}
\usepackage[T1]{fontenc}
\usepackage{xcolor}
\usepackage{graphicx}
\usepackage{geometry}
\usepackage[italian]{babel}
\usepackage{tabularx}
\usepackage{longtable}
\usepackage{hyperref}
\usepackage[bottom]{footmisc}
\usepackage{fancyhdr}
\usepackage{titlesec}
\setcounter{secnumdepth}{4}
\usepackage{amsmath, amssymb}
\usepackage{array}
\usepackage{graphicx}
\usepackage{layouts}
\usepackage{url}
\usepackage{comment}
\usepackage{eurosym}


\begin{document}

	\thispagestyle{empty}
	\begin{titlepage}
		\begin{center}
			\includegraphics[scale = 0.05]{../../Res/logo_unipd.png}\\
			\bigskip
			\large \textbf{Università degli Studi di Padova} \\
			\vfill
			\includegraphics[scale = 0.7]{../../Res/BugPharma_Logo.png}\\
			\huge \textbf{Gruppo Bug Pharma} \\
			\vfill
			\large \texttt{bugpharma10@gmail.com}
			\vfill
			\Huge \textbf{Verbale interno del 02-11-2021}\\
			
			\large
			
			\vfill
			\textbf{Ordine del giorno:} \\
			Individuazione aspetti positivi e negativi capitolato C2
			\vfill
			
			\begin{tabular}{r|l}
				\textbf{Approvazione} &  -\\
				\textbf{Redazione} &  \parbox[t]{3.5cm}{Andrea Salmaso \\Nicla Faccioli}\\
				\textbf{Verifica} &  Silvia Giro\\
				\textbf{Stato} & Verificato \\
				\textbf{Uso} & Interno
			\end{tabular}
			\vfill
			
		\end{center}
	\end{titlepage}

	\newpage

	\section{Informazioni generali}
	\subsection{Luogo e data dell'incontro}
	\begin{itemize}
		\item \textbf{Luogo}: videoconferenza Zoom;
		\item \textbf{Data}: 02-11-2021;
		\item \textbf{Ora di inizio}: 13:30;
		\item \textbf{Ora di fine}: 14:15.
	\end{itemize}
	
	\subsection{Presenze}
	\begin{itemize}
		\item \textbf{Totale presenze}: 3;
		\item \textbf{Presenti}:
		\begin{itemize}
			\item Lorenzo Piran (segretario);
			\item Michele Masetto;
			\item Silvia Giro.
		\end{itemize}
		\item \textbf{Assenti}:
		\begin{itemize}
			\item Nicla Faccioli;
			\item Sara Nanni;
			\item Andrea Salmaso;
			\item Nicholas Sertori.
		\end{itemize}
		%\item \textbf{Partecipanti esterni}:
		%\begin{itemize}
		%	\item Gregorio Piccoli (Zucchetti S.p.A.).
		%\end{itemize}
	\end{itemize}

	\newpage

	\section{Punti salienti}
		Sono emersi i seguenti pro e contro, già in parte discussi nell'incontro di Mercoledì 20 Ottobre 2021:
		\begin{itemize}
			\item PRO:
			\begin{itemize}
				\item Il progetto permetterebbe al gruppo di cimentarsi in una sfida impegnativa, ma spendibile in futuro
				in termini di conoscenze acquisite;
			\end{itemize}
			\item CONTRO:
				\begin{itemize}
				\item I componenti del gruppo hanno esperienza quasi nulla sugli argomenti chiave che dovranno essere
				trattati, fatto che potrebbe portare a diversi inconvenienti.
				\item Sono emerse le seguenti domande, le cui risposte potrebbero portare ad un numero eccessivamente
				alto di casi d'uso, di possibile difficile gestione:
				\begin{enumerate}
					\item Cosa succederebbe se ci fosse un pacco smarrito (es: problema di logistica, anche se il venditore
					ha inviato il pacco e l'acquirente ha pagato - soldi bloccati nello smart contract)?;
					\item Serve anonimato? È possibile ricontattare il venditore, per natura stessa della blockchain?;
					\item Che politica si applica per resi e rimborsi? E come si può esser sicuri che il venditore, una volta
					ricevuto il pagamento, non scappi con i soldi? (L'acquirente paga ma il pacco non contiene quello che dovrebbe -
					Problema del mattone);
					\item Come confermare la ricezione? (Ad esempio, se si utilizza la scansione del \texttt{QR code}, cosa
					succederebbe se all'arrivo del pacco l'acquirente non avesse la possibilità di fare lo scan per qualche motivo);
					\item E se si volesse delegare qualcuno per ritirare il pacco?;
					\item È fattibile scegliere questo capitolato senza avere conoscenze pregresse sulle tecnologie che
					sembrano essere un po' il punto centrale?
				\end{enumerate}
			\end{itemize}
		\end{itemize}
\end{document}
