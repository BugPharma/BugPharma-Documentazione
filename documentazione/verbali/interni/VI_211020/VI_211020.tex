\documentclass[11pt]{article}

\usepackage[english,italian]{babel}
\usepackage[a4paper, top=2cm, bottom=1.5cm, left=2cm, right=2cm]{geometry}
\usepackage{float}
\usepackage{ltablex}
\usepackage{titling}
\usepackage{blindtext}
\usepackage[utf8]{inputenc}
\usepackage[T1]{fontenc}
\usepackage{xcolor}
\usepackage{graphicx}
\usepackage{geometry}
\usepackage[italian]{babel}
\usepackage{tabularx}
\usepackage{longtable}
\usepackage{hyperref}
\usepackage[bottom]{footmisc}
\usepackage{fancyhdr}
\usepackage{titlesec}
\setcounter{secnumdepth}{4}
\usepackage{amsmath, amssymb}
\usepackage{array}
\usepackage{graphicx}
\usepackage{layouts}
\usepackage{url}
\usepackage{comment}
\usepackage{eurosym}


\begin{document}

	\thispagestyle{empty}
	\begin{titlepage}
		\begin{center}
			\includegraphics[scale = 0.05]{../../Res/logo_unipd.png}\\
			\bigskip
			\large \textbf{Università degli Studi di Padova} \\
			\vfill
			\includegraphics[scale = 0.7]{../../Res/BugPharma_Logo.png}\\
			\huge \textbf{Gruppo Bug Pharma} \\
			\vfill
			\large \texttt{bugpharma10@gmail.com}
			\vfill
			\Huge \textbf{Verbale interno del 20-10-2021}\\
			
			\large
			
			\vfill
			\textbf{Ordine del giorno:} \\
			Lettura capitolati di interesse e discussione nome e logo del gruppo
			\vfill
			
			\begin{tabular}{r|l}
				\textbf{Approvazione} &  Nicholas Sertori\\
				\textbf{Redazione} &  \parbox[t]{3.5cm}{Andrea Salmaso \\Nicla Faccioli}\\
				\textbf{Verifica} &  Silvia Giro\\
				\textbf{Stato} & Approvato \\
				\textbf{Uso} & Interno
			\end{tabular}
			\vfill
			
		\end{center}
	\end{titlepage}

	\newpage

	\section{Informazioni generali}
	\subsection{Luogo e data dell'incontro}
	\begin{itemize}
		\item \textbf{Luogo}: videoconferenza Zoom;
		\item \textbf{Data}: 20-10-2021;
		\item \textbf{Ora di inizio}: 16:00;
		\item \textbf{Ora di fine}: 16:50.
	\end{itemize}
	
	\subsection{Presenze}
	\begin{itemize}
		\item \textbf{Totale presenze}: 6 su 7;
		\item \textbf{Presenti}:
		\begin{itemize}
			\item Lorenzo Piran; 
			\item Michele Masetto;
			\item Silvia Giro;
			\item Nicla Faccioli;
			\item Sara Nanni;
			\item Andrea Salmaso.
			
		\end{itemize}
		\item \textbf{Assenti}:
			\begin{itemize}
				\item Nicholas Sertori.
			\end{itemize}
		%\item \textbf{Partecipanti esterni}:
	\end{itemize}
	
	\newpage
	
	\section{Punti salienti}
		\begin{itemize}
			\item Tutti i membri del gruppo hanno già letto, individualmente, i capitolati usciti;
			\item Sono stati esclusi i capitolati C3 e C6;
			\item I capitolati di maggior interesse risultano essere, in ordine:
			\begin{itemize}
				\item C5 - Login Warrior;
				\item C2 - BlockChange;
				\item C1 - Bot4Me.
			\end{itemize}
			\item Si è deciso di concentrarsi inizialmente sui primi due (C5 e C2);
			\item Si è deciso di suddividersi in due gruppi per analizzare i primi due capitolati scelti ed evidenziare
			aspetti positivi/negativi ed eventuali domande da porre alle aziende, con la seguente modalità:
			\begin{enumerate}
				\item Rilettura individuale dei capitolati assegnati;
				\item Brainstorming tra i membri dei due sottogruppi.
			\end{enumerate}
			I due sottogruppi sono composti come segue:
			\begin{itemize}
				\item C5: Andrea, Nicholas, Sara, Nicla;
				\item C2: Lorenzo, Michele, Silvia.
			\end{itemize}
			
			\item Sono stati proposti alcuni possibili nomi per il gruppo:
				\begin{itemize}
					\item 1ncept1on Code;
					\item Bug Pharma.
				\end{itemize}
			\item Si è deciso di poter proporre, in settimana, altri nomi e loghi e di votare i definitivi in un secondo momento
			con l'ausilio di \texttt{Google Form}.
		\end{itemize}	
\end{document}
