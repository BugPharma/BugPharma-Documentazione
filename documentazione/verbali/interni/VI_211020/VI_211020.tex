\documentclass[11pt]{article}

\usepackage[english,italian]{babel}
\usepackage[a4paper, top=2cm, bottom=1.5cm, left=2cm, right=2cm]{geometry}
\usepackage{float}
\usepackage{ltablex}
\usepackage{titling}
\usepackage{blindtext}
\usepackage[utf8]{inputenc}
\usepackage[T1]{fontenc}
\usepackage{xcolor}
\usepackage{graphicx}
\usepackage{geometry}
\usepackage[italian]{babel}
\usepackage{tabularx}
\usepackage{longtable}
\usepackage{hyperref}
\usepackage[bottom]{footmisc}
\usepackage{fancyhdr}
\usepackage{titlesec}
\setcounter{secnumdepth}{4}
\usepackage{amsmath, amssymb}
\usepackage{array}
\usepackage{graphicx}
\usepackage{layouts}
\usepackage{url}
\usepackage{comment}
\usepackage{eurosym}






\begin{document}

	\thispagestyle{empty}
	\begin{titlepage}
		\begin{center}
			\includegraphics[scale = 0.05]{/home/nicla/Scrivania/SWE/verbali/logo_unipd.png}\\
			\large \textbf{Università degli Studi di Padova} \\
			\vfill
			\includegraphics[scale = 0.7]{/home/nicla/Scrivania/SWE/verbali/logo_small.jpg}\\
			\large \textbf{Bug Pharma} \\
			\vfill
			\large
			E-mail: 
			\texttt{bugpharma10@gmail.com}
			\vfill
			\Huge \textbf{Verbale interno del 20-10-2021}\\
			
			\large
			%per verbali esterni
			%\textbf{Synclab - Blockchange}\\ 
			
			%per verbali interni
			\vfill
			\textbf{Ordine del giorno:} \\
			Lettura dei capitolati di interesse e discussione di nome e logo del gruppo
			
			
			\vfill
			
			
			\begin{tabular}{r|l}
				\textbf{Approvazione} &  -\\
				\textbf{Redazione} &  \parbox[t]{5cm}{Andrea Salmaso \\Nicla Faccioli}\\
				\textbf{Verifica} &  -\\
				\textbf{Stato} & Redatto \\
				\textbf{Uso} & Interno
			\end{tabular}
			\vfill
			
		\end{center}
	\end{titlepage}

	\section{Informazioni generali}
	\subsection{Luogo e data dell'incontro}
	\begin{itemize}
		\item \textbf{luogo}: videoconferenza Zoom;
		\item \textbf{data}: 20-10-2021;
		\item \textbf{ora di inizio}: 16:00;
		\item \textbf{ora di fine}: 16:50.
	\end{itemize}
	
	\subsection{Presenze}
	\begin{itemize}
		\item \textbf{Totale presenze}: 6 su 7;
		\item \textbf{Presenti}:
		\begin{itemize}
			\item Lorenzo Piran; 
			\item Michele Masetto;
			\item Silvia Giro;
			\item Nicla Faccioli (segretaria);
			\item Sara Nanni;
			\item Andrea Salmaso.
			
		\end{itemize}
		\item \textbf{Assenti}:
			\begin{itemize}
				\item Nicholas Sertori.
			\end{itemize}
		%\item \textbf{Partecipanti esterni}:
	\end{itemize}
	
	\newpage
	
	\section{Punti salienti}
		\begin{itemize}
			\item tutti hanno già letto i capitolati usciti individualmente;
			\item capitolati esclusi: C3, C6;
			\item Capitolati di nostro interesse, in ordine:
			\begin{itemize}
				\item C5 - Login Warrior;
				\item C2 - BlockChange;
				\item C1 - Bot4Me;
			\end{itemize}
			\item Per ora ci concentriamo sui primi due;
			\item Suddivisione in due gruppi per analizzare i primi due capitolati scelti ed evidenziare aspetti positivi/ negativi ed eventuali domande da porre all’azienda:
				\begin{itemize}
					\item C5: Andrea, Nicholas, Sara, Nicla;
					\item C2: Lorenzo, Michele, Silvia.
				\end{itemize}
			\item Iniziamo rileggendo con attenzione i capitolati assegnati individualmente e poi facciamo brainstorming in due sottogruppi;
			\item possibili nomi per il gruppo:
				\begin{itemize}
					\item 1ncept1on Code
					\item Bug Pharma
				\end{itemize}
			\item In settimana proponiamo nomi e loghi e decidiamo con google form.
		\end{itemize}	
\end{document}