\documentclass[11pt]{article}

\usepackage[english,italian]{babel}
\usepackage[a4paper, top=2cm, bottom=1.5cm, left=2cm, right=2cm]{geometry}
\usepackage{float}
\usepackage{ltablex}
\usepackage{titling}
\usepackage{blindtext}
\usepackage[utf8]{inputenc}
\usepackage[T1]{fontenc}
\usepackage{xcolor}
\usepackage{graphicx}
\usepackage{geometry}
\usepackage[italian]{babel}
\usepackage{tabularx}
\usepackage{longtable}
\usepackage{hyperref}
\usepackage[bottom]{footmisc}
\usepackage{fancyhdr}
\usepackage{titlesec}
\setcounter{secnumdepth}{4}
\usepackage{amsmath, amssymb}
\usepackage{array}
\usepackage{graphicx}
\usepackage{layouts}
\usepackage{url}
\usepackage{comment}
\usepackage{eurosym}






\begin{document}

	\thispagestyle{empty}
	\begin{titlepage}
		\begin{center}
			\includegraphics[scale = 0.05]{logo_unipd.png}\\
			\large \textbf{Università degli Studi di Padova} \\
			\vfill
			\includegraphics[scale = 0.7]{logo_small.jpg}\\
			\large \textbf{Bug Pharma} \\
			\vfill
			\large
			E-mail: 
			\texttt{bugpharma10@gmail.com}
			\vfill
			\Huge \textbf{Verbale interno del 02-11-2021}\\
			
			\large
			%per verbali esterni
			%\textbf{Zucchetti S.p.A. - Login Warrior}\\
			
			%per verbali interni
			\vfill
			\textbf{Ordine del giorno:} \\
			Redazione domande in vista di possibile incontro con Zucchetti S.p.A.
			
			
			\vfill
			
			
			\begin{tabular}{r|l}
				\textbf{Approvazione} &  -\\
				\textbf{Redazione} &  \parbox[t]{5cm}{Andrea Salmaso \\Nicla Faccioli}\\
				\textbf{Verifica} &  -\\
				\textbf{Stato} & Redatto \\
				\textbf{Uso} & Interno
			\end{tabular}
			\vfill
			
		\end{center}
	\end{titlepage}

	\section{Informazioni generali}
	\subsection{Luogo e data dell'incontro}
	\begin{itemize}
		\item \textbf{Luogo}: videoconferenza Zoom;
		\item \textbf{Data}: 02-11-2021;
		\item \textbf{Ora di inizio}: 14:30;
		\item \textbf{Ora di fine}: 15:30.
	\end{itemize}
	
	\subsection{Presenze}
	\begin{itemize}
		\item \textbf{Totale presenze}: 4;
		\item \textbf{Presenti}:
		\begin{itemize}
			\item Nicla Faccioli (segretaria);
			\item Sara Nanni;
			\item Andrea Salmaso;
			\item Nicholas Sertori.
		\end{itemize}
		\item \textbf{Assenti}:
		\begin{itemize}
			\item Lorenzo Piran;
			\item Michele Masetto;
			\item Silvia Giro.
		\end{itemize}
		%\item \textbf{Partecipanti esterni}:
		%\begin{itemize}
		%	\item Gregorio Piccoli (Zucchetti S.p.A.).
		%\end{itemize}
	\end{itemize}

	\newpage

	\section{Punti salienti}
		\begin{itemize}
			\item Sono emerse le seguenti domande da porre al primo incontro con l'azienda Zucchetti S.p.A.:
			\begin{enumerate}
				\item Che tipo di sito serve?
				C'è altro oltre alla visualizzazione dei dati, come ad esempio il caricamento dei dati da parte dell'utente?
				Che livello di interazione con il sito serve da parte dell'utente?
				\item Qual è il punto centrale del progetto?
				Solo la preparazione e l'analisi dei dati o dobbiamo preoccuparci anche della sicurezza?
				\item Che tipo di dati ci sono nei file \texttt{CSV}? Consigli su come utilizzarli?
				\item Quali sono i parametri per giudicare un login sospetto?
				\item Che tipo di formato preferiscono per i manuali da consegnare (utilizzo ed espansione)?
				\item Consigli generali relativi agli anni precedenti?
			\end{enumerate}
			\item Il capitolato risulta essere interessante e fattibile in base alle conoscenze pregresse e agli
			obiettivi dei membri del gruppo.
		\end{itemize}
		
\end{document}