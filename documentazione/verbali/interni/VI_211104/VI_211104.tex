\documentclass[11pt]{article}

\usepackage[english,italian]{babel}
\usepackage[a4paper, top=2cm, bottom=1.5cm, left=2cm, right=2cm]{geometry}
\usepackage{float}
\usepackage{ltablex}
\usepackage{titling}
\usepackage{blindtext}
\usepackage[utf8]{inputenc}
\usepackage[T1]{fontenc}
\usepackage{xcolor}
\usepackage{graphicx}
\usepackage{geometry}
\usepackage[italian]{babel}
\usepackage{tabularx}
\usepackage{longtable}
\usepackage{hyperref}
\usepackage[bottom]{footmisc}
\usepackage{fancyhdr}
\usepackage{titlesec}
\setcounter{secnumdepth}{4}
\usepackage{amsmath, amssymb}
\usepackage{array}
\usepackage{graphicx}
\usepackage{layouts}
\usepackage{url}
\usepackage{comment}
\usepackage{eurosym}


\begin{document}

	\thispagestyle{empty}
	\begin{titlepage}
		\begin{center}
			\includegraphics[scale = 0.05]{../../Res/logo_unipd.png}\\
			\bigskip
			\large \textbf{Università degli Studi di Padova} \\
			\vfill
			\includegraphics[scale = 0.7]{../../Res/BugPharma_Logo.png}\\
			\huge \textbf{Gruppo Bug Pharma} \\
			\vfill
			\large \texttt{bugpharma10@gmail.com}
			\vfill
			\Huge \textbf{Verbale interno del 04-11-2021}\\
			
			\large
			
			\vfill
			\textbf{Ordine del giorno:} \\
			Punto della situazione
			\vfill
			
			\begin{tabular}{r|l}
				\textbf{Approvazione} &  -\\
				\textbf{Redazione} &  \parbox[t]{3.5cm}{Andrea Salmaso \\Nicla Faccioli}\\
				\textbf{Verifica} &  Silvia Giro\\
				\textbf{Stato} & Verificato \\
				\textbf{Uso} & Interno
			\end{tabular}
			\vfill
			
		\end{center}
	\end{titlepage}

	\newpage

	\section{Informazioni generali}
	\subsection{Luogo e data dell'incontro}
	\begin{itemize}
		\item \textbf{Luogo}: Aula 1A150 - Dipartimento di Matematica Tullio Levi-Civita;
		\item \textbf{Data}: 04-11-2021;
		\item \textbf{Ora di inizio}: 12:00;
		\item \textbf{Ora di fine}: 12:30.
	\end{itemize}
	
	\subsection{Presenze}
	\begin{itemize}
		\item \textbf{Totale presenze}: 4 su 7;
		\item \textbf{Presenti}:
		\begin{itemize}  
			\item Michele Masetto;
			\item Nicla Faccioli;
			\item Sara Nanni;
			\item Andrea Salmaso.
		\end{itemize}
		\item \textbf{Assenti}:
			\begin{itemize}
				\item Lorenzo Piran;
				\item Silvia Giro;
				\item Nicholas Sertori.
			\end{itemize}
		%\item \textbf{Partecipanti esterni}:
	\end{itemize}

	\newpage

	\section{Punti salienti}
		\begin{itemize}
			\item È stata stilata la lista dei documenti da consegnare per la candidatura:
				\begin{itemize}
					\item Verbali degli incontri interni e con i proponenti;
					\item Motivazioni della scelta;
					\item Preventivo costi $\rightarrow$ ore di lavoro per ciascun ruolo;
					\item Data promessa di consegna (piano di lavoro con scadenze approssimative);
					\item Totale ore produttive/impegni individuali.
				\end{itemize}
			\item Sono stati proposti alcuni strumenti di possibile utilità, i quali, potenzialmente, entreranno a far parte
			del Way of Working:
				\begin{itemize}
					\item \texttt{gitHub}/\texttt{gitLab};
					\item \texttt{Google Docs};
					\item \texttt{LaTex}: per la versione finale dei documenti;
					\item Strumento per tickets/issues: \texttt{gitHub}, \texttt{Jira} o simili;
					\item Strumento per diagrammi di Gantt : \texttt{Instagantt} (a pagamento) o simili;
					\item Strumento per la creazione automatica di grafici e tabelle per tenere traccia delle differenze
					tra preventivi e consuntivi (es: \texttt{Microsoft Excel}).
				\end{itemize}
		\end{itemize}
		
		
		
		
		
		
		
		
		
		
		
		
		
		
		
\end{document}
