\documentclass[11pt]{article}

\usepackage[english,italian]{babel}
\usepackage[a4paper, top=2cm, bottom=1.5cm, left=2cm, right=2cm]{geometry}
\usepackage{float}
\usepackage{ltablex}
\usepackage{titling}
\usepackage{blindtext}
\usepackage[utf8]{inputenc}
\usepackage[T1]{fontenc}
\usepackage{xcolor}
\usepackage{graphicx}
\usepackage{geometry}
\usepackage[italian]{babel}
\usepackage{tabularx}
\usepackage{longtable}
\usepackage{hyperref}
\usepackage[bottom]{footmisc}
\usepackage{fancyhdr}
\usepackage{titlesec}
\setcounter{secnumdepth}{4}
\usepackage{amsmath, amssymb}
\usepackage{array}
\usepackage{graphicx}
\usepackage{layouts}
\usepackage{url}
\usepackage{comment}
\usepackage{eurosym}






\begin{document}

	\thispagestyle{empty}
	\begin{titlepage}
		\begin{center}
			\includegraphics[scale = 0.05]{/home/nicla/Scrivania/SWE/verbali/logo_unipd.png}\\
			\large \textbf{Università degli Studi di Padova} \\
			\vfill
			\includegraphics[scale = 0.7]{/home/nicla/Scrivania/SWE/verbali/logo_small.jpg}\\
			\large \textbf{Bug Pharma} \\
			\vfill
			\large
			E-mail: 
			\texttt{bugpharma10@gmail.com}
			\vfill
			\Huge \textbf{Verbale interno del 04-11-2021}\\
			
			\large
			%per verbali esterni
			%\textbf{Synclab - Blockchange}\\ 
			
			%per verbali interni
			\vfill
			\textbf{Ordine del giorno:} \\
			Punto della situazione
			
			
			\vfill
			
			
			\begin{tabular}{r|l}
				\textbf{Approvazione} &  -\\
				\textbf{Redazione} &  \parbox[t]{5cm}{Andrea Salmaso \\Nicla Faccioli}\\
				\textbf{Verifica} &  -\\
				\textbf{Stato} & Redatto \\
				\textbf{Uso} & Interno
			\end{tabular}
			\vfill
			
		\end{center}
	\end{titlepage}

	\section{Informazioni generali}
	\subsection{Luogo e data dell'incontro}
	\begin{itemize}
		\item \textbf{luogo}: aula 1A150 - Dipartimento di Matematica Tullio Levi-Civita ;
		\item \textbf{data}: 04-11-2021;
		\item \textbf{ora di inizio}: 12:00;
		\item \textbf{ora di fine}: 12:30.
	\end{itemize}
	
	\subsection{Presenze}
	\begin{itemize}
		\item \textbf{Totale presenze}: 4 su 7;
		\item \textbf{Presenti}:
		\begin{itemize}  
			\item Michele Masetto;
			\item Nicla Faccioli (segretaria);
			\item Sara Nanni;
			\item Andrea Salmaso.
		\end{itemize}
		\item \textbf{Assenti}:
			\begin{itemize}
				\item Lorenzo Piran;
				\item Silvia Giro;
				\item Nicholas Sertori.
			\end{itemize}
		%\item \textbf{Partecipanti esterni}:
	\end{itemize}

	\newpage

	\section{Punti salienti}
		\begin{itemize}
			\item I documenti da consegnare per la candidatura sono:
				\begin{itemize}
					\item verbali degli incontri interni e con i proponenti;
					\item motivazioni della scelta;
					\item preventivo costi → ore di lavoro per ciascun ruolo;
					\item data promessa di consegna (piano di lavoro con scadenze approssimative);
					\item Totale ore produttive/ impegni individuali.
				\end{itemize}
			\item Way of working: proposte di strumenti possibili da utilizzare:
				\begin{itemize}
					\item gitHub/ gitLab;
					\item Google Docs;
					\item LaTex: per la versione finale dei documenti;
					\item strumento per tickets/issues : github / jira o simili;
					\item strumento per diagrammi di gantt : instagantt (a pagamento) o simili;
					\item strumento per la creazione automatica di grafici e tabelle per tenere traccia delle differenze tra preventivi e consuntivi (es: Excel).
				\end{itemize}
		\end{itemize}
		
		
		
		
		
		
		
		
		
		
		
		
		
		
		
\end{document}